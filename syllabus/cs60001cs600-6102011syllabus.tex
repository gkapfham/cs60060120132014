\input syllabuspre
\begin{document}
\MYTITLE{Syllabus}
\MYHEADERS{Syllabus}{}

\subsection*{Course Instructor}
Dr.\ Gregory M.\ Kapfhammer\\
\noindent Office Location: Alden Hall 108 \\
\noindent Office Phone: +1 814-332-2880 \\
\noindent Email: \url{gkapfham@allegheny.edu} \\
\noindent Twitter: \url{@GregKapfhammer} \\
\noindent Web Site: \url{http://www.cs.allegheny.edu/sites/gkapfham/} 

\subsection*{Instructor's Office Hours}

\begin{itemize}
	\itemsep 0em
	\item Monday: 1:00 pm -- 2:30 pm (30 minute time slots)
	\item Tuesday: 2:30 pm -- 4:00 pm (15 minute time slots)
	\item Wednesday: 4:30 pm -- 5:00 pm (15 minute time slots)
	\item Thursday: 9:00 am -- 10:00 am (15 minute time slots) {\em and} \\ \hspace*{.69in} 2:30 pm -- 4:00 pm (15 minute time slots)
	\item Friday: 1:00 pm -- 2:30 pm (10 minute time slots) {\em and} \\ \hspace*{.49in} 4:30 pm -- 5:00 pm (5 minute time slots)
\end{itemize}

\noindent
To schedule a meeting with me during my office hours, please visit my Web site and click the ``Schedule'' link in the
top right-hand corner. Now, you can browse my office hours or schedule an appointment by clicking the correct link and
then reserving an open time slot.  Please ask Dr.\ Jumadinova and Dr.\ Roos about their availability during the Fall
2013 and Spring 2014 semesters.

\subsection*{Schedule}

{\bf CMPSC 600}
\begin{center}
\begin{tabular}{r|l}
\hline
September 6 & Submit request for first and second reader \\ 
November 4 -- November 8 & Register for CMPSC 610 with first reader \\
November 4 -- December 10 & Oral defense of thesis proposal \\
December 16 & Submit two chapters to course instructor by 5 pm\\
\hline
Before September 20 & Schedule weekly meeting time with your first reader \\
Before October 16 & Schedule proposal defense with Pauline Lanzine \\
Before November 4 & Proposal formally approved by first reader \\
Before November 28 & Get technical report number from Pauline Lanzine\\
\hline
Entire academic semester & Friday class session, 2:30 pm -- 3:20 pm \\
September through December & Meet with first reader on a regular basis \\
\hline
\end{tabular}
\end{center}

% NOTE: Senior grades due March 7, 2013

{\bf CMPSC 610}
\begin{center}
\begin{tabular}{r|l}
\hline
Before January 24 & Schedule weekly meeting time with your first reader \\ 
Before March 3 & Schedule thesis defense with Pauline Lanzine \\
April 2 & Submit unbound and signed thesis to first and second readers \\
April 7 -- April 25 & Complete oral defense of senior thesis \\
May 1 & Submit final bound and signed thesis to Pauline Lanzine by 4 pm\\
\hline
Entire academic semester & Meet with first reader on a regular basis \\ 
\hline
\end{tabular}
\end{center}

\noindent
The schedule for CMPSC 600 and 610 may change as the course instructor deems appropriate.

\vspace*{-.1in}
\subsection*{Grading}

Final grades are determined by vote of the entire faculty of the Department of Computer Science, not just the course instructor
for CMPSC 600 or CMPSC 610.

Your grade in CMPSC 600 will be based on a combination of the following activities and deliverables. Percentages are not
given because we recognize that the senior thesis experience differs from one student to the next and that there are many
variables, such as the nature of the project and the availability of external resources, that can influence the relative
importance of these criteria. However, it is important to note that a large percentage of your grade depends upon your
written proposal, your proposal defense, and your two chapters.  

% Using the aforementioned deadlines as a framework, work
% with your first reader to establish a schedule of deadlines appropriate to your project and then adhere to this
% schedule.
% 

\vspace*{-.05in}
\begin{center}
\begin{tabular}{lp{4in}}

\bf Participation & This includes meeting regularly with your first reader. Although the exact details about frequency and
length of each meeting must be established with your first reader, you should adhere to the previously stated schedule. \\

\bf Written proposal & This document must be approved by your senior thesis adviser and formatted according to the 
department's thesis proposal style requirements, available from the course Web site. \\

\bf Proposal defense & This event is scheduled in consultation with your first and second reader and the building
coordinator, Pauline Lanzine. \\

\bf Chapters & Any two chapters of your final senior thesis must be submitted to the course instructor by the
aforementioned deadline.  The chapters must be formatted in the department's thesis style; please note that
this style is available from the course Web site and different from the proposal's style.

\end{tabular}
\end{center}

\vspace*{-.15in}
\noindent
Your grade for CMPSC 610 will be based on a combination of these activities and deliverables:
\begin{center}
\begin{tabular}{lp{4in}}
\bf Participation & While exact details about the frequency and length of meetings must be established with your first
reader, you should meet regularly and adhere to the previously stated schedule. \\ 

\bf Written thesis & This document must be formatted according to the department's thesis style guide, available from
the course Web site. \\

\bf Thesis defense & This event is scheduled in consultation with your first and second reader and the building
coordinator, Pauline Lanzine. \\

% May be scheduled only with first reader's approval; the building coordinator, Pauline Lanzine in
% room 110, will schedule the defense for you once you have obtained permission from your first reader.
% 
\end{tabular}
\end{center}

\subsection*{Course Expectations and Deliverables --- CMPSC 600}

\medskip
\noindent{\bf Participation.} Once your readers have been assigned, you must regularly meet with your first reader, who
will report on your participation when the department's faculty meet to assign grades.  Students are expected to come to
each meeting with a status update on their progress and a meeting agenda.  Students should conclude each meeting by
listing the tasks that they want to complete before the next meeting. Evidence of regular meetings should be submitted
to the course instructor. 

\medskip
\noindent{\bf Proposal.}
The proposal should follow the department's proposal style and thus must include an abstract, the main body of your
proposal, a tentative schedule for completing the project, a bibliography, and any other information deemed important by
your first reader. This will often include one or more of the following: a survey of the existing literature; an overview of
your proposed technique; the description of an evaluation method; examples or code artifacts or evidence that you
understand the nature of the work you are proposing and can feasibly complete it in the time available. 
The proposal must adhere to professional standards of writing. Your first reader will make suggestions on your submitted
drafts; students are expected to revise multiple proposal drafts.

% You must adhere
% to professional standards of writing---
% 
% your first reader will make suggestions on your submitted drafts about how to
% improve your proposal.  Expect to do a number of revisions.
% 

You must work at a pace that will ensure that your first reader can formally approve the final draft of you thesis
proposal before the stated deadline.  Failure to secure formal approval of your thesis proposal before this date will
result in the reduction of your final grade in CMPSC 600.

% Failure to finish this task before the deadline will result in the reduction of your final grade in CMPSC 600.

\medskip
\noindent{\bf Proposal Defense.}
A proposal defense is a prepared, formal presentation of about ten minutes in which you lay out the essential pieces of
your chosen project under the assumption that your first and second reader have studied your proposal.  Following the
presentation that is supported by polished slides, you will participate in a discussion with your readers to identify
potential problems, refine or modify some aspects of the project proposal, and ensure that your project is feasible
and appropriate. All aspects of your proposal defense should be prepared in consultation with your first reader.
You must schedule your proposal defense before the stated deadline. Your grade in CMPSC 600 will be reduced if you miss the
deadline for scheduling or conducting your defense.

% This will be
% followed by a discussion with your readers to identify 
% 
% potential problems, refine or modify some aspects of the project
% proposal, and insure that your project is a feasible and appropriate one. {\em You should prepare your proposal defense
% in cooperation with your first reader.}
% 

\medskip
\noindent{\bf Thesis Chapters.} Your two chapters, due on the previously stated date, should represent a
significant addition to or extension of the material in your proposal. Don't simply ``split the proposal into two
chapters'' --- this usually does not work well since your chapters represent work completed, not work being proposed.
Chapters are judged according to the same professional standards as the proposal; they must include a full bibliography, a 
preliminary table of contents, lists of any figures and tables, and any other items required by your first reader.

As you write your chapters in consultation with your first and second reader, allow these individuals to comment on a
draft and then make all of the requested changes.  Plan to write several drafts of the chapters before submitting them
on the due date; failure to submit the chapters by the stated deadline will result in the reduction of your final grade
in CMPSC 600.

\subsection*{Course Expectations and Deliverables --- CMPSC 610}

%\medskip
%{\bf Be sure to sign up for the section of CMPSC 610 that is
%taught by your first reader.}

\medskip
\noindent{\bf Participation.} You must regularly meet with your first reader, who will report on your participation when
the department's faculty meet to assign grades.  Students are expected to come to each meeting with a status update on
their progress and a meeting agenda.  Students should conclude each meeting by listing the tasks that they want to
complete before the next meeting. Evidence of regular meetings should be submitted to the course instructor at the end
of the semester. 

% {\bf It is your obligation} to regularly meet with your first reader,
% who will report on your participation when the Department faculty
% meet to assign grades.
% 

	% This allows the month of April for you to schedule  and present a thesis defense, make any required changes to your
	% thesis, and submit final copies for binding no later than the end of the final exam period (and preferably much
	% earlier). 


\medskip
\noindent{\bf Written Thesis.} In consultation with your first reader and in accordance with the stated deadlines, you
must work out a schedule for completion of your thesis research and your written document. All senior theses are due,
properly formatted and signed (but not bound), on the stated due date. 	Your thesis should both follow the department's
style and adhere to professional standards of writing. You are expected to work closely with your first reader in the
creation of this document.  Your grade in CMPSC 610 will be reduced if you fail to submit your unbound thesis on time.

Following your defense, you must submit the bound copy of your senior thesis by the aforementioned due date.  This
document must incorporate any changes that were requested by your first and second reader. Seniors who have not
delivered the signed and appropriately bound copies of their senior thesis by the stated deadline will receive an
incomplete and will not graduate.

\medskip 
\noindent{\bf Thesis Defense.} The standards for this presentation are the same as for the proposal defense ---  you
must give a ten minute presentation supported by polished slides and adhere to all of the other stated requirements.
Part of your grade will depend on how well you are able to discuss aspects of your thesis, including implications of
your work, connections between your research and other areas of computer science, and possible extensions or
improvements of your research ideas.  You are expected to work with your first reader in preparing your oral defense.
Your grade in CMPSC 610 will be reduced if you do not schedule or conduct your thesis defense by the stated deadlines.

% \noindent{\bf Late Policy} 
% Failure to meet the deadlines above will incur a late penalty to be decided by
% the faculty of the Department of Computer Science.
% 
% The following policy was adopted by
%the entire computer science department, effective beginning in fall of 2004:
%\begin{quote}
%All assignments will have a %given 
%due date.  The %assignment is to be
%products of your work are to be turned in {\bf at the beginning of class or
%lab} on the due date (or as specified by your instructor).
%% turned in at the beginning of the class on that due date.  
%Late
%assignments will be accepted for up to one week past the assigned due
%date with a 15\% penalty.  All late assignments must be submitted
%%the beginning of the 
%%first 
%%class
%%or laboratory
%%that is scheduled 
%within one week after the given due date.
%No assignments will be accepted for credit after the one week late period.
%It is the student's responsibility to keep secure backups of all assignments
%and labs.
%\end{quote}

% \subsection*{Email}
% \begin{quote}
% ``The use of email is a primary method of communication on campus. \ldots
% All students are provided with a campus email account and address while
% enrolled at Allegheny and are expected to check the account on a regular
% basis.'' [{\em The Compass}, the Allegheny College student handbook]
% \end{quote}
% I will sometimes need to send out announcements to the class with
% things such as clarifications, changes in the
% schedule, or other matters. I will use your Allegheny College e-mail account
% to do this. It is your responsibility to check your e-mail at least once a
% day, and to make certain that your e-mail is working correctly (able to send
% and receive messages).
% 

\vspace*{-.2in}
\subsection*{Email}

Using your Allegheny College email address, I will sometimes send out class announcements about matters such as
assignment clarifications or changes in the schedule. It is your responsibility to check your email at least once a day
and to ensure that you can reliably send and receive emails. This class policy is based on the following statement in
{\em The Compass}, the college's student handbook.

\vspace*{-.1in}
\begin{quote}
``The use of email is a primary method of communication on campus. \ldots
All students are provided with a campus email account and address while
enrolled at Allegheny and are expected to check the account on a regular
basis.'' 
\end{quote}
\vspace*{-.3in}

\subsection*{Disability Services}

The Americans with Disabilities Act (ADA) is a federal anti-discrimination statute that provides comprehensive civil
rights protection for persons with disabilities.  Among other things, this legislation requires all students with
disabilities be guaranteed a learning environment that provides for reasonable accommodation of their disabilities.
Students with disabilities who believe they may need accommodations in this class are encouraged to contact Disability
Services at 332-2898.  Disability Services is part of the Learning Commons and is located in Pelletier Library.
Please do this as soon as possible to ensure that approved accommodations are implemented in a timely fashion.

\subsection*{Honor Code}

The Academic Honor Program that governs the entire academic program at Allegheny College is described in the Allegheny
Course Catalogue.  The Honor Program applies to all work that is submitted for academic credit or to meet non-credit
requirements for graduation at Allegheny College.  This includes all work assigned for these classes (e.g., source code,
technical diagrams, and your written thesis).  All students who have enrolled in the College will work under the Honor
Program.  Each student who has matriculated at the College has acknowledged the following pledge:

\vspace*{-.1in}
\begin{quote}
I hereby recognize and pledge to fulfill my responsibilities, as defined in the Honor Code, and to maintain the
integrity of both myself and the College community as a whole.  
\end{quote}
\vspace*{-.15in}

\subsection*{Welcome to an Adventure in Computer Science}

CMPSC 600 and 610 afford you the opportunity to pursue independent research in computer science.  Moreover, these
courses properly position you to conduct ground-breaking work that can have a positive influence on your future career
and graduate school prospects, the students and faculty in the Department of Computer Science, the Allegheny College
community, and a broader society that heavily relies on computer hardware and software.  At the start of your senior
year, I invite you to pursue these two classes with great enthusiasm and vigor.

% \subsection*{Special Needs and Disabilities}
% The Americans with Disabilities Act (ADA) is a federal anti-discrimination
% statute that provides comprehensive civil rights protection for persons
% with disabilities. Among other things, this legislation requires that
% all students with disabilities be guaranteed a learning environment
% that provides for reasonable accommodation of their disabilities.
% If you believe  you have a disability requiring an accommodation,
% please contact the Learning Commons at 332-2898.
% 
\end{document}
