\input syllabuspre
\begin{document}
\MYTITLE{Syllabus}
\MYHEADERS{Syllabus}{}

\subsection*{Course Instructor}
Dr.\ Gregory M.\ Kapfhammer\\
\noindent Office Location: Alden Hall 108 \\
\noindent Office Phone: +1 814-332-2880 \\
\noindent Email: \url{gkapfham@allegheny.edu} \\
\noindent Twitter: \url{@GregKapfhammer} \\
\noindent Web Site: \url{http://www.cs.allegheny.edu/sites/gkapfham/} 

\subsection*{Instructor's Office Hours}

\begin{itemize}
	\itemsep 0em
	\item Monday: 1:00 pm -- 2:30 pm (30 minute time slots)
	\item Tuesday: 2:30 pm -- 4:00 pm (15 minute time slots)
	\item Wednesday: 4:30 pm -- 5:00 pm (15 minute time slots)
	\item Thursday: 9:00 am -- 10:00 am (15 minute time slots) {\em and} \\ \hspace*{.69in} 2:30 pm -- 4:00 pm (15 minute time slots)
	\item Friday: 1:00 pm -- 2:30 pm (10 minute time slots) {\em and} \\ \hspace*{.49in} 4:30 pm -- 5:00 pm (5 minute time slots)
\end{itemize}

\noindent
To schedule a meeting with me during my office hours, please visit my Web site and click the ``Schedule'' link in the
top right-hand corner. Now, you can browse my office hours or schedule an appointment by clicking the correct link and
then reserving an open time slot.  Please ask Dr.\ Jumadinova and Dr.\ Roos about their availability during the Fall
2013 and Spring 2014 semesters.

\subsection*{Schedule}

{\bf CMPSC 600}
\begin{center}
\begin{tabular}{r|l}
\hline
September 6 & Submit request for first and second reader \\ 
November 4 -- November 8 & Register for CMPSC 610 with first reader \\
November 4 -- December 10 & Oral defense of thesis proposal \\
December 16 & Submit two chapters to course instructor \\
\hline
Before October 16 & Schedule proposal defense with Pauline Lanzine \\
Before November 4 & Proposal formally approved by first reader \\
Before November 28 & Get technical report number from Pauline Lanzine\\
\hline
September through December & Friday class session, 2:30 pm -- 3:20 pm \\
September through December & Meet with first reader on a regular basis \\
\hline
\end{tabular}
\end{center}

% NOTE: Senior grades due March 7, 2013

{\bf CMPSC 600}
\begin{center}
\begin{tabular}{r|l}
\hline
Before March 3 & Schedule thesis defense with Pauline Lanzine \\
April 2 & Submit unbound thesis to first and second readers \\
April 7 -- April 25 & Oral defense of senior thesis \\
May 1 at 9 am & Submit final bound and signed thesis to Pauline Lanzine \\
\hline
Entire academic semester & Meet with first reader on a regular basis \\ 
\hline
\end{tabular}
\end{center}

\subsection*{Grading}

Final grades are determined by vote of
the entire faculty of the Department of Computer Science, not just the
instructor of CMPSC 600 or CMPSC 610.

Your grade in {\bf CMPSC 600} will be based on some combination of the following
activities and deliverables. Percentages are not given because we
recognize that the senior project experience differs from one student to the
next and there are many variables (nature of the project, obtaining
permission to use resources from other researchers, etc.) that can
influence the relative importance of these criteria. However, it should be
obvious that a large percentage of your grade depends upon
your written proposal, your proposal defense (quality of 
the presentation, ability to answer questions, etc.), and your two chapters.
Work with your first
reader to establish a schedule of deadlines appropriate 
to your project and then adhere to this schedule.
\begin{center}
\begin{tabular}{lp{4in}}
\bf Participation & This includes meeting regularly with your first
reader; exact details about frequency, times, etc.\ must be established with 
your first reader\\

\bf Written proposal & Must be approved by your senior thesis advisor and
formatted according to Department thesis proposal style requirements
(available at the course home page listed above).\\

\bf Proposal defense & May be scheduled only with first reader's approval;
the building coordinator, Pauline Lanzine in room 110, 
will schedule the defense for you once you have obtained
permission from your first reader.\\

\bf Chapters & Two chapters (needn't be the first two) of 
your final senior thesis must be handed in
by the last day of classes. 
They must be formatted according to
the Department's thesis style (not the same as proposal style --- available
at the course home page listed above). 

\end{tabular}
\end{center}

Your grade for {\bf CMPSC 610} will be based on 
some combination of the following activities and deliverables:
\begin{center}
\begin{tabular}{lp{4in}}
\bf Participation & This includes meeting regularly with your first
reader; exact details about frequency, times, etc.\ should be established with 
your first reader\\

\bf Written thesis & Must be
formatted according to Department thesis style requirements (available at the
course home page listed above).\\

\bf Thesis defense & May be scheduled only with first reader's approval;
the building coordinator, Pauline Lanzine in room 110, 
will schedule the defense for you once you have obtained
permission from your first reader.

\end{tabular}
\end{center}

\subsection*{Course Expectations and Deliverables --- CMPSC 600}

\medskip
\noindent{\bf Participation.}
Once your readers have been assigned,
{\bf it is your obligation} to regularly meet with your first reader,
who will report on your participation when the Department faculty
meet to assign grades.

\medskip
\noindent{\bf Proposal.}
The proposal  must follow the Department proposal style and must include
an abstract, a bibliography, a tentative schedule for completing the
project, and any other information deemed important by your first
reader. This will often include one or more of the following:
a survey of existing literature;
a description of experiments to be performed;
examples or code artifacts or evidence that you understand the
nature of the work you are proposing and can feasibly complete it
in the time available. You must adhere to professional standards of
writing---your first reader will make suggestions on your submitted
drafts about how to improve your proposal.
Expect to do a number of revisions.

Try to work at a pace that will allow the proposal to be approved by your 
first reader no later than {\bf Monday, 7 November}. 
This allows sufficient time to schedule your proposal defense during
the month of November and complete your chapters by the last week of class.

\medskip
\noindent{\bf Proposal Defense.}
It is your responsibility to contact the building coordinator, Pauline
Lanzine, to schedule your proposal defense
once you have had your proposal approved by your first reader.

A proposal defense is a prepared, formal presentation of about five to
seven minutes in which you lay out the essential pieces of your proposal
(your first and second reader will have already seen your written
proposal). This will be followed by a discussion with your readers to
identify potential problems, refine or modify some aspects of the
project proposal, and insure that your project is a feasible and
appropriate one. {\em You should prepare your proposal defense in
cooperation with your first reader.}

\medskip
\noindent{\bf Chapters and Bibliography.} Your two chapters, due the last week 
of classes, should represent 
a significant addition to or extension of the material
in your proposal. Don't simply ``split the proposal into two chapters'' ---
this usually does not work well since
your chapters 
represent work completed, not work being proposed. Chapters are judged
according to the same professional standards as the proposal; they must
include a full bibliography.

You should write your chapters in cooperation with your first reader,
letting the reader comment on drafts of the chapters.

\subsection*{Course Expectations and Deliverables --- CMPSC 610}

%\medskip
%{\bf Be sure to sign up for the section of CMPSC 610 that is
%taught by your first reader.}

\medskip
\noindent{\bf Participation.}
{\bf It is your obligation} to regularly meet with your first reader,
who will report on your participation when the Department faculty
meet to assign grades.

\medskip
\noindent{\bf Written Thesis.}
You will work with your first (and possibly your second) reader to
work out a schedule for completion of your thesis research and your
written document. All senior theses are due, properly formatted and signed, 
but {\em not} bound, no later than {\bf Monday, 2 April} (one week after the end
of Spring Break). This allows the month of April for you to schedule  and
present a thesis defense, make any required changes to your thesis, and
submit final copies for binding no later than the end of the final
exam period (and preferably much earlier). {\em Any senior who has not 
delivered appropriately bound copies of his or her thesis by the morning
that senior grades are due will receive an incomplete and will not graduate.}

Your thesis should follow Department style and should adhere to
professional standards of writing. You are expected to work closely with
your first reader in the creation of this document.

\medskip 
\noindent{\bf Thesis Defense.}
The standards for presentation are the same as for the proposal defense.
Part of your grade will depend on how well you are able to discuss
aspects of your thesis, including 
implications of your work and possible extensions or improvements.
You are expected to work with your first reader in preparing your
oral defense.

\noindent{\bf Late Policy.} 
Failure to meet the deadlines above will incur a late penalty to be decided by
the faculty of the Department of Computer Science.

% The following policy was adopted by
%the entire computer science department, effective beginning in fall of 2004:
%\begin{quote}
%All assignments will have a %given 
%due date.  The %assignment is to be
%products of your work are to be turned in {\bf at the beginning of class or
%lab} on the due date (or as specified by your instructor).
%% turned in at the beginning of the class on that due date.  
%Late
%assignments will be accepted for up to one week past the assigned due
%date with a 15\% penalty.  All late assignments must be submitted
%%the beginning of the 
%%first 
%%class
%%or laboratory
%%that is scheduled 
%within one week after the given due date.
%No assignments will be accepted for credit after the one week late period.
%It is the student's responsibility to keep secure backups of all assignments
%and labs.
%\end{quote}

\subsection*{Email}
\begin{quote}
``The use of email is a primary method of communication on campus. \ldots
All students are provided with a campus email account and address while
enrolled at Allegheny and are expected to check the account on a regular
basis.'' [{\em The Compass}, the Allegheny College student handbook]
\end{quote}
I will sometimes need to send out announcements to the class with
things such as clarifications, changes in the
schedule, or other matters. I will use your Allegheny College e-mail account
to do this. It is your responsibility to check your e-mail at least once a
day, and to make certain that your e-mail is working correctly (able to send
and receive messages).

\subsection*{Special Needs and Disabilities}
The Americans with Disabilities Act (ADA) is a federal anti-discrimination
statute that provides comprehensive civil rights protection for persons
with disabilities. Among other things, this legislation requires that
all students with disabilities be guaranteed a learning environment
that provides for reasonable accommodation of their disabilities.
If you believe  you have a disability requiring an accommodation,
please contact the Learning Commons at 332-2898.

\end{document}
